\usepackage{geometry}

\usepackage{amsmath}
\usepackage{amssymb,amsfonts,textcomp}
\usepackage[T1]{fontenc}
\usepackage[utf8]{inputenc}
\usepackage{times} % Times New Roman font
\usepackage{setspace}
\usepackage[pdftex]{graphicx}

% \usepackage{listings}
% \lstset{
%   basicstyle=\fontsize{10}{1.15},
%   basicstyle=\fontfamily{pcr}\selectfont,
%   breaklines=true, % Allow breaking lines
%   tabsize=2, % Tab size
%   showstringspaces=false, % Don't show spaces in strings
%   %frame=single, % Add a frame around the lstlisting content
% }

\usepackage{xcolor}
\usepackage{listings}

\lstdefinestyle{pythonstyle}{
    language=Python,
    backgroundcolor=\color{white},   % set background color
    basicstyle=\ttfamily\footnotesize, % set font and size
    keywordstyle=\color{blue},        % color of keywords
    commentstyle=\color{green},       % color of comments
    stringstyle=\color{red},          % color of strings
    numbers=left,                    % show line numbers
    numberstyle=\tiny\color{gray},    % style of line numbers
    stepnumber=1,                    % number every line
    numbersep=5pt,                   % distance of line numbers from code
    showstringspaces=false,           % don't show spaces in strings
    breaklines=true,                 % break lines if too long
    breakatwhitespace=true,          % break lines at whitespace
    tabsize=4,                       % set tab size
}

\lstset{style=pythonstyle}  % Apply the style globally


% Set line spacing to 1.5
\setstretch{1.5}

\geometry{a4paper, portrait, margin=0.7in, nohead, nofoot}

\usepackage{titlesec}
\titleformat{\section}[block]{\normalfont\large\bfseries}{\thesection}{1em}{}
\titleformat{\subsection}[block]{\normalfont\large\bfseries}{\thesubsection}{1em}{}

\newcommand{\Num}{}
\newcommand{\Topic}{}

\newcommand{\labno}[1]{%
    \renewcommand{\Num}{#1}%
}

\newcommand{\labtopic}[1]{%
    \renewcommand{\Topic}{#1}%
}
